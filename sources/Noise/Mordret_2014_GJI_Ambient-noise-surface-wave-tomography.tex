%! TeX root = ../../*.tex

\currentpaper[https://doi.org/10.1093/gji/ggu217]{mordret2014ambient}

\begin{frame}[c]{\titleprefix: Misfit of dispersion curves}

  \centering

  \begin{minipage}{0.5\textwidth}
    \secincgraph[0.20]{Mordret_2014_GJI_Misfit-of-dispersion-curves.png}
  \end{minipage}
  %
  \begin{minipage}{0.40\textwidth}
    \tiny

    To \emph{avoid to over-fit the dispersion curves},
    the misfit is the area of the theoretical dispersion curve outside the area
    defined by the measured dispersion curve and its uncertainties,
    normalized by the area of the measured dispersion curve.

    \begin{figureblock}{Figure 2}
      Illustration of the misfit computation between a theoretical dispersion
      curve in brown and a measured dispersion curve with its uncertainties
      in pink.
      In this example, the misfit is the normalised area
      $ (\text{dS1} + \text{dS2}) / S $.
    \end{figureblock}
  \end{minipage}

\end{frame}

