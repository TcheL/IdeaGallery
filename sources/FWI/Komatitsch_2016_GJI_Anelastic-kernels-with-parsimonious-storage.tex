%! TeX root = ../../*.tex

\currentpaper[https://doi.org/10.1093/gji/ggw224]{komatitsch2016anelastic}

\begin{frame}[c]{\titleprefix: Storage strategy for anelastic kernels}

  \centering

  \begin{minipage}{0.4\textwidth}
    \secincgraph[0.14]{Komatitsch_2016_GJI_Storage-strategy-%
      for-anelastic-kernels.png}
  \end{minipage}
  %
  \hspace{1 cm}
  %
  \begin{minipage}{0.5\textwidth}
    \tiny
    \begin{figureblock}{Figure 2}
      (a) store the entire forward run to disk, read it back from disk in
      reverse order while computing the adjoint wavefield:
      \emph{the required amount of storage, and heavy I/O.}

      (b) perform the forward run first and store its final time step to disk,
      and in a second stage perform the adjoint run while simultaneously
      redoing the forward run backwards, reversing time and starting from
      the stored final time step:
      \emph{because of going backwards, unstable in the anelastic case or
      in the present of energy loss.}

      (c) during the first stage, save checkpointing/restart files to disk
      every few hundred or thousand time steps; during the second stage,
      preform two runs simultaneously, and perform the forward run in chunks
      in reverse order but in the forward direction:
      \emph{always stable, and also exact; meaningful only if storage buffer
      is large enough.}
    \end{figureblock}
  \end{minipage}

\end{frame}
