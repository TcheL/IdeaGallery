
\currentpaper[https://doi.org/10.1111/j.1365-246X.2004.02453.x]%
{tromp2005seismic}

\begin{frame}[c]{\titleprefix: Adjoint method}

  \centering

  \begin{minipage}{0.5\textwidth}
    \secincgraph[0.19]{Tromp_2005_GJI_Adjoint-method.png}
  \end{minipage}
  %
  \hspace{0.5cm}
  %
  \begin{minipage}{0.42\textwidth}
    \tiny
    \begin{figureblock}{Figure 3}
      Sequence of interactions between the regular and adjoint $SH$ wavefields
      during the construction of the banana-doughnut kernel $\bar K_{\beta}$.
      This particular $\bar K_{\beta}$ kernel is for $SH_S$, i.e.
      the $SH~\beta$ kernel obtained by time-reversing the $S$ arrival.
      Each row represents an instantaneous interaction between the regular and
      adjoint fields.
      From the left column to the right column are shown the regular field,
      the adjoint field, the interaction field and the instantaneous sensitivity
      to shear velocity perturbations, $\bar K_{\beta}$.
      The $\bar K_{\beta}$ kernel is constructed by integrating the interaction
      field, shown in the third column, over time.
      The source is labelled by the \faStarO{} and
      the receiver by the $\square$.
    \end{figureblock}
  \end{minipage}

\end{frame}

