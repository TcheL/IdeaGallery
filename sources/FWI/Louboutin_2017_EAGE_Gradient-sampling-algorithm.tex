
\currentpaper[https://slim.gatech.edu/Publications/Public/Conferences/%
SINBAD/2015/Fall/louboutin2015SINBADFess/louboutin2015SINBADFess.pdf]%
{louboutin2017extending}

\begin{frame}[c]{\titleprefix: Gradient sampling algorithm}

  \centering

  \begin{minipage}{0.3\textwidth}
    \secincgraph[0.15]{Louboutin_2017_EAGE_Gradient-sampling-step.png}
  \end{minipage}
  %
  \hspace{1 cm}
  %
  \begin{minipage}{0.5\textwidth}
    \tiny
    \begin{figureblock}{Gradient smapling algorithm}
      \begin{itemize}
        \tiny
        \item sampling $N + 1$ vector $\bf x_{ki}$ in a ball
          $B_{\epsilon_k} (\bf x_k)$ defined as all $\bf x_{ki}$ such that
          $|| \bf x_k - \bf x_{ki} ||^2_2 < \epsilon_k$,
          where $\epsilon_k$ is the maximum distance between the current
          estimate and a sampled vector;
        \item calculating gradients for each sample, i.e.,
          $\bf g_{ki} = \nabla \Phi (\bf x_{ki})$;
        \item computing descent directions as a weighted sum over all sampled
          gradient, i.e.,
          \[ \bf g_k \approx \sum_{i = 0}^p \omega_i \bf g_{ki},
             \text{ such that } \sum_{i = 0}^p \omega_i = 1,
             \omega_i > 0 \forall i \]
        \item updating the model according to $\bf x_{k + 1} = \bf x_k - \alpha
          \bf H^{ - 1} \bf g_k$, where $\alpha$ is a step length
          obtained from a line search and $\bf H^{ - 1}$ is an approximation
          of the inverse Hessian.
      \end{itemize}
    \end{figureblock}
  \end{minipage}

\end{frame}
